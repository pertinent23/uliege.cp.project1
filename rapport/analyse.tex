% !TEX root = ./main.tex
%%%%%%%%%%%%%%%%%%%%%%%%%%%%%%%%%%%%%%%%%%%%%%%%%%%%%%%%%%%%%%%%%%%%%%%%%%%%%%%%%%%%%%%%%%
% Dans ce fichier, vous devez définir (Input/Output/O.U.) proprement et clairement le    %
% problème.
%
% Il est aussi demandé de réaliser une analyse complète (i.e., découpe en SPs)           %
%%%%%%%%%%%%%%%%%%%%%%%%%%%%%%%%%%%%%%%%%%%%%%%%%%%%%%%%%%%%%%%%%%%%%%%%%%%%%%%%%%%%%%%%%%

\section{Définition et Analyse du Problème}\label{analyse}
%%%%%%%%%%%%%%%%%%%%%%%%%%%%%%%%%%%%%%%%%%%%

\subsection{Input:}

\begingroup
Le nombre d'éléments à analyser:
\endgroup
\begin{lstlisting}
int N;
\end{lstlisting}

\vspace{0.3cm}

\begingroup
Un tableau d'entier de:
\endgroup
\begin{lstlisting}
int T[N];
\end{lstlisting}

\subsection{Output:}
\quad 
cette fonction retourne la somme des éléments du tableau entre la position du minimum \emph{$min\_pos$} et la position 
du maximun \emph{$max\_pos$}.
\par
Ainsi soient:
$a=\min(min\_pos, max\_pos) \land b=\max(min\_pos, max\_pos)$
\par
$somme=\sum_{i=a}^{b}(T[i]) \land 0 \le a \le b < N$

\subsection{Objets Utilisés:}

\begingroup
$min\_pos$: Contient l'indice de $\min{T}$
\endgroup
\begin{lstlisting}
int min_pos;
\end{lstlisting}

\begingroup
$max\_pos$: Contient l'indice de $\max{T}$
\endgroup
\begin{lstlisting}
int max_pos;
\end{lstlisting}

\begingroup
$resultat$: Contient la somme des élément du tableau $\max{T}$
\endgroup
\begin{lstlisting}
int resultat;
\end{lstlisting}

\subsection{Sous-Problème:}
Vu que le l'objectif de la fonction est d'écrire un programme de complexité $O(N)$,
on aura un seul sous-problème qui va consister en:
\begin{itemize}
    \item Determiner la position du maximun
    \item Determiner la position du minimun
    \item Calculer la somme des éléments du tableau entre le maximun et le minimum
\end{itemize}