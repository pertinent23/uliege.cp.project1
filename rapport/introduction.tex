% !TEX root = ./main.tex
%%%%%%%%%%%%%%%%%%%%%%%%%%%%%%%%%%%%%%%%%%%%%%%%%%%%%%%%%%%%%%%%%%%%%%%%%%%%%%%%%%%%%%%%%%
% Rédigez ici l'introduction de votre rapport.                                           %
%%%%%%%%%%%%%%%%%%%%%%%%%%%%%%%%%%%%%%%%%%%%%%%%%%%%%%%%%%%%%%%%%%%%%%%%%%%%%%%%%%%%%%%%%%
\section{Introduction}\label{introduction}
%%%%%%%%%%%%%%%%%%%%%%%

\par
Nous avons un tableau \textbf{T[N]} de \textbf{N} éléments \textbf{entiers} dans lequel nous souhaitons déterminer le minimun et
le maximun. Ainsi calculer la somme des éléments entre le minimun et le maximun. 
\par

La solution évidente consiste à determiner le minimun et la maximun dans un premier temps
puis de calculer la somme entre ces derniers.
\par

Cependant la solution que nous voulons obtenir doit avoir une complexité de O\(N\) dans 
la pire des cas. 
\par

Le travail que nous allons mener, consiste à formaliser le, problème, produire un invariant 
graphique puis un invariant formel, et enfin produire le code, montrer qu'il fonctionne et 
prouver que sa complexité est en O\(N\).