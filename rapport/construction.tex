% !TEX root = ./main.tex
%%%%%%%%%%%%%%%%%%%%%%%%%%%%%%%%%%%%%%%%%%%%%%%%%%%%%%%%%%%%%%%%%%%%%%%%%%%%%%%%%%%%%%%%%%
% Dans cette section, il est demandé d'appliquer l'approche constructive pour la         %
% construction de votre code.                                                            %
%                                                                                        %
% Pour chaque Sous-Problème (une sous-section/SP):                                       %
% - {Pré} INIT {INV}                                                                     %
% - déterminer le Critère d'Arrêt (et donc le Gardien de Boucle)                         %
% - {INV & B} ITER {INV}                                                                 %
% - {INV & !B} END {Post}                                                                %
% - Fonction de Terminaison (pensez à justifier sur base de l'Invariant Graphique)       %
% (une sous-sous-section/tiret)                                                          %
%%%%%%%%%%%%%%%%%%%%%%%%%%%%%%%%%%%%%%%%%%%%%%%%%%%%%%%%%%%%%%%%%%%%%%%%%%%%%%%%%%%%%%%%%%
\section{Approche Constructive}
%%%%%%%%%%%%%%%%%%%%%%%%%%%%%%%%

\begin{lstlisting}[caption={somme.c}]
//$N=N_0>0 \land T = T_0$
int somme(int *T, int N,  int *min_pos, int *max_pos)
{
    //$N=N_0>0 \land T = T_0$
    assert(N>0 && T != NULL && min_pos != NULL && max_pos != NULL);
    //$N=N_0>0 \land T = T_0$

    int resultat, tmp1, tmp2, i;
    //$N=N_0>0 \land T = T_0$

    if (N==1) 
        //$N=N_0>0 \land T = T_0 \land T[min\_pos]=\min{T} \land T[max\_pos]=\max{T}$
        return T[*min_pos = *max_pos = 0];
    
    //$N=N_0>0 \land T = T_0$
    if (T[0] > T[1])
        *min_pos = 1 + (*max_pos = 0);
        //$N=N_0>0 \land T = T_0 \land T[min\_pos]=\min{T} \land T[max\_pos]=\max{T}$
    else
        *max_pos = 1 + (*min_pos = 0);
        //$N=N_0>0 \land T = T_0 \land T[min\_pos]=\min{T} \land T[max\_pos]=\max{T}$
    
    //$\rightarrow N=N_0>0 \land T = T_0 \land T[min\_pos]=\min{T} \land T[max\_pos]=\max{T}$

    resultat = T[0] + T[1];
    //$N=N_0>0 \land T = T_0 \land T[min\_pos]=\min{T} \land T[max\_pos]=\max{T} \land resultat = \sum_{k=min\_pos}^{max\_pos}(T[k])$
    tmp1 = T[0];
    //$tmp1=\sum_{}^{\min(min\_pos, max\_pos)}(T[.])$
    //$N=N_0>0 \land T = T_0 \land T[min\_pos]=\min{T} \land T[max\_pos]=\max{T} \land resultat = \sum_{k=min\_pos}^{max\_pos}(T[k])$
    tmp2 = 0;
    //$tmp2=\sum_{\max(min\_pos, max\_pos)}^{N-1}(T[.])$
    //$N=N_0>0 \land T = T_0 \land T[min\_pos]=\min{T} \land T[max\_pos]=\max{T} \land resultat = \sum_{k=min\_pos}^{max\_pos}(T[k])$
    i = 2;
    //$INV \overset{}{\equiv} N=N_0>0 \land T = T_0 \land T[min\_pos]=\min{T} \land T[max\_pos]=\max{T} $
    //$\land resultat = \sum_{k=min\_pos}^{max\_pos}(T[k]) \land 1 \le i \le N$

    while(i<N) 
    {
        //$INV \overset{}{\equiv} N=N_0>0 \land T = T_0 \land T[min\_pos]=\min{T} \land T[max\_pos]=\max{T} $
        //$\land resultat = \sum_{k=min\_pos}^{max\_pos}(T[k]) \land 1 \le i \le N-1$
        if(T[i]<T[*min_pos])
        {
            if (*min_pos<*max_pos) 
                resultat -= tmp1;
                //$resultat = \sum_{k=min\_pos}^{max\_pos}(T[k]$
            
            //$T[min\_pos]=\min{T}$
            *min_pos = i;
            //$T[min\_pos]=\min{T}$
            tmp1 = resultat;
            //$tmp1=\sum_{}^{\min(min\_pos, max\_pos)}(T[.])$
            //$tmp2=\sum_{\max(min\_pos, max\_pos)}^{i-1}(T[.])$
            tmp2 = 0;
            //$tmp2=\sum_{\max(min\_pos, max\_pos)}^{i}(T[.])$
        }
        else if(T[i]>T[*max_pos])
        {
            if (*max_pos<*min_pos) 
                resultat -= tmp1;
                //$resultat = \sum_{k=min\_pos}^{max\_pos}(T[k]$
            
            //$T[max\_pos]=\max{T}$
            *max_pos = i;
            //$T[max\_pos]=\max{T}$
            tmp1 = resultat;
            //$tmp1=\sum_{}^{\min(min\_pos, max\_pos)}(T[.])$
            //$tmp2=\sum_{\max(min\_pos, max\_pos)}^{i-1}(T[.])$
            tmp2 = 0;
            //$tmp2=\sum_{\max(min\_pos, max\_pos)}^{i}(T[.])$
        }
        else 
        {
            //$tmp2=\sum_{\max(min\_pos, max\_pos)}^{i-1}(T[.])$
            tmp2 += T[i];
            //$tmp2=\sum_{\max(min\_pos, max\_pos)}^{i}(T[.])$
        }
        
        resultat += T[i];
        //$resultat = \sum_{k=min\_pos}^{max\_pos}(T[k]$
        //$1 \le i \le N-1$
        i++;
        //$1 \le i \le N$
    }

    //$INV \overset{}{\equiv} N=N_0>0 \land T = T_0 \land T[min\_pos]=\min{T} \land T[max\_pos]=\max{T} $
    //$\land resultat = \sum_{k=max\_pos}^{N-1}(T[k]) \land 1 \le i \le N$

    return resultat - tmp2;
    //$T=T_0 \land 0 \le min\_pos \le max\_pos \le N-1 \land T[min\_pos]=\min{T} \land T[max\_pos]=\max{T}$
}
//$T=T_0 \land 0 \le min\_pos \le max\_pos \le N-1 \land T[min\_pos]=\min{T} \land T[max\_pos]=\max{T}$
\end{lstlisting}

%Il est possible de faire référence à la ligne \ref{code:ret} de l'extrait de code.
