%%%%%%%%%%%%%%%%%%%%%%%%%%%%%%%%%%%%%%%%%%%%%%%%%%%%%%%%%%%%%%%%%%%%%%%%%%%%%%%%%%%%%%%%%%
% Ceci est le fichier principal du template template à utiliser pour les rapports du     %
% projet 1 (Construction de Programme) d'INFO0947.                                       %
%                                                                                        %
% Vous devez décommenter et compléter les commandes introduites plus bas (intitule, ...) %
% avant de pouvoir compiler le fichier LaTeX.  Pensez à configurer votre Makefile en     %
% conséquence.                                                                           %
%                                                                                        %
% Le contenu et la structure du rapport sont imposés.  Vous devez compléter les          %
% différents fichiers .tex inclus dans ce fichier avec votre production.                 %
%%%%%%%%%%%%%%%%%%%%%%%%%%%%%%%%%%%%%%%%%%%%%%%%%%%%%%%%%%%%%%%%%%%%%%%%%%%%%%%%%%%%%%%%%%

% !TEX root = ./main.tex
% !TEX engine = latexmk -pdf
% !TEX buildOnSave = true
\documentclass[a4paper, 11pt, oneside]{article}

\usepackage[utf8]{inputenc}
\usepackage[T1]{fontenc}
\usepackage[french]{babel}
\usepackage{array}
\usepackage{shortvrb}
\usepackage{listings}
\usepackage[fleqn]{amsmath}
\usepackage{amsfonts}
\usepackage{fullpage}
\usepackage{enumerate}
\usepackage{graphicx}             % import, scale, and rotate graphics
\usepackage{subfigure}            % group figures
\usepackage{alltt}
\usepackage{url}
\usepackage{indentfirst}
\usepackage{eurosym}
\usepackage{listings}
\usepackage{color}
\usepackage[table,xcdraw,dvipsnames]{xcolor}

% Change le nom par défaut des listing
\renewcommand{\lstlistingname}{Extrait de Code}


\definecolor{mygray}{rgb}{0.5,0.5,0.5}
\newcommand{\coms}[1]{\textcolor{MidnightBlue}{#1}}

\lstset{
    language=C, % Utilisation du langage C
    commentstyle={\color{MidnightBlue}}, % Couleur des commentaires
    frame=single, % Entoure le code d'un joli cadre
    rulecolor=\color{black}, % Couleur de la ligne qui forme le cadre
    stringstyle=\color{RawSienna}, % Couleur des chaines de caractères
    numbers=left, % Ajoute une numérotation des lignes à gauche
    numbersep=5pt, % Distance entre les numérots de lignes et le code
    numberstyle=\tiny\color{mygray}, % Couleur des numéros de lignes
    basicstyle=\tt\footnotesize,
    tabsize=3, % Largeur des tabulations par défaut
    keywordstyle=\tt\bf\footnotesize\color{Sepia}, % Style des mots-clés
    extendedchars=true,
    captionpos=b, % sets the caption-position to bottom
    texcl=true, % Commentaires sur une ligne interprétés en Latex
    showstringspaces=false, % Ne montre pas les espace dans les chaines de caractères
    escapeinside={(>}{<)}, % Permet de mettre du latex entre des <( et )>.
    inputencoding=utf8,
    literate=
  {á}{{\'a}}1 {é}{{\'e}}1 {í}{{\'i}}1 {ó}{{\'o}}1 {ú}{{\'u}}1
  {Á}{{\'A}}1 {É}{{\'E}}1 {Í}{{\'I}}1 {Ó}{{\'O}}1 {Ú}{{\'U}}1
  {à}{{\`a}}1 {è}{{\`e}}1 {ì}{{\`i}}1 {ò}{{\`o}}1 {ù}{{\`u}}1
  {À}{{\`A}}1 {È}{{\`E}}1 {Ì}{{\`I}}1 {Ò}{{\`O}}1 {Ù}{{\`U}}1
  {ä}{{\"a}}1 {ë}{{\"e}}1 {ï}{{\"i}}1 {ö}{{\"o}}1 {ü}{{\"u}}1
  {Ä}{{\"A}}1 {Ë}{{\"E}}1 {Ï}{{\"I}}1 {Ö}{{\"O}}1 {Ü}{{\"U}}1
  {â}{{\^a}}1 {ê}{{\^e}}1 {î}{{\^i}}1 {ô}{{\^o}}1 {û}{{\^u}}1
  {Â}{{\^A}}1 {Ê}{{\^E}}1 {Î}{{\^I}}1 {Ô}{{\^O}}1 {Û}{{\^U}}1
  {œ}{{\oe}}1 {Œ}{{\OE}}1 {æ}{{\ae}}1 {Æ}{{\AE}}1 {ß}{{\ss}}1
  {ű}{{\H{u}}}1 {Ű}{{\H{U}}}1 {ő}{{\H{o}}}1 {Ő}{{\H{O}}}1
  {ç}{{\c c}}1 {Ç}{{\c C}}1 {ø}{{\o}}1 {å}{{\r a}}1 {Å}{{\r A}}1
  {€}{{\euro}}1 {£}{{\pounds}}1 {«}{{\guillemotleft}}1
  {»}{{\guillemotright}}1 {ñ}{{\~n}}1 {Ñ}{{\~N}}1 {¿}{{?`}}1
}
\newcommand{\tablemat}{~}

%%%%%%%%%%%%%%%%% TITRE %%%%%%%%%%%%%%%%
% Complétez et décommentez les définitions de macros suivantes :
% \newcommand{\intitule}{Le titre du travail}
% \newcommand{\GrNbr}{1742}
% \newcommand{\PrenomUN}{Galileo}
% \newcommand{\NomUN}{Galilei}
% \newcommand{\PrenomDEUX}{Octave}
% \newcommand{\NomDEUX}{Urbain}

\renewcommand{\tablemat}{\tableofcontents}

%%%%%%%% ZONE PROTÉGÉE : MODIFIEZ UNE DES DIX PROCHAINES %%%%%%%%
%%%%%%%%            LIGNES POUR PERDRE 2 PTS.            %%%%%%%%
\title{INFO0947: \intitule}
\author{Groupe \GrNbr : \PrenomUN~\textsc{\NomUN}, \PrenomDEUX~\textsc{\NomDEUX}}
\date{}
\begin{document}

\maketitle
\newpage
\tablemat
\newpage

%%%%%%%%%%%%%%%% RAPPORT %%%%%%%%%%%%%%%

% Inclusion des différentes sections

% !TEX root = ./main.tex
%%%%%%%%%%%%%%%%%%%%%%%%%%%%%%%%%%%%%%%%%%%%%%%%%%%%%%%%%%%%%%%%%%%%%%%%%%%%%%%%%%%%%%%%%%
% Rédigez ici l'introduction de votre rapport.                                           %
%%%%%%%%%%%%%%%%%%%%%%%%%%%%%%%%%%%%%%%%%%%%%%%%%%%%%%%%%%%%%%%%%%%%%%%%%%%%%%%%%%%%%%%%%%
\section{Introduction}\label{introduction}
%%%%%%%%%%%%%%%%%%%%%%%


% !TEX root = ./main.tex
%%%%%%%%%%%%%%%%%%%%%%%%%%%%%%%%%%%%%%%%%%%%%%%%%%%%%%%%%%%%%%%%%%%%%%%%%%%%%%%%%%%%%%%%%%
% Dans cette section, introduisez toutes les notations mathématiques que vous jugez      %
% utiles à la réalisation du projet.                                                     %
%%%%%%%%%%%%%%%%%%%%%%%%%%%%%%%%%%%%%%%%%%%%%%%%%%%%%%%%%%%%%%%%%%%%%%%%%%%%%%%%%%%%%%%%%%
\section{Formalisation du Problème}\label{formalisation}
%%%%%%%%%%%%%%%%%%%%%%%%%%%%%%%%%%%

\subsection{Utilisez les bons opérateurs}

Voir la table \ref{table:op}.

\begin{table}[!h]
\centering
\begin{tabular}{l c}
Nom & Op \\
\hline
ET & $\land$ \\
OU & $\lor$ \\
Quantification universelle & $\forall$ \\
Quantification existentielle & $\exists$ \\
\end{tabular}
\caption{Opérateurs les plus usuels en logique}
\label{table:op}
\end{table}

\subsection{Trouver un symbole précis}

Voir ce site : \url{http://detexify.kirelabs.org/classify.html}. Il suffit de dessiner le symbole dont vous avez besoin et le site trouvera (normalement) la bonne commande à taper (ainsi que le package à éventuellement inclure si besoin est).


% !TEX root = ./main.tex
%%%%%%%%%%%%%%%%%%%%%%%%%%%%%%%%%%%%%%%%%%%%%%%%%%%%%%%%%%%%%%%%%%%%%%%%%%%%%%%%%%%%%%%%%%
% Dans ce fichier, vous devez définir (Input/Output/O.U.) proprement et clairement le    %
% problème.
%
% Il est aussi demandé de réaliser une analyse complète (i.e., découpe en SPs)           %
%%%%%%%%%%%%%%%%%%%%%%%%%%%%%%%%%%%%%%%%%%%%%%%%%%%%%%%%%%%%%%%%%%%%%%%%%%%%%%%%%%%%%%%%%%

\section{Définition et Analyse du Problème}\label{analyse}
%%%%%%%%%%%%%%%%%%%%%%%%%%%%%%%%%%%%%%%%%%%%

\subsection{Input:}

\begingroup
Le nombre d'éléments à analyser:
\endgroup
\begin{lstlisting}
int N;
\end{lstlisting}

\vspace{0.3cm}

\begingroup
Un tableau d'entier de:
\endgroup
\begin{lstlisting}
int T[N];
\end{lstlisting}

\subsection{Output:}
\quad 
cette fonction retourne la somme des éléments du tableau entre la position du minimum \emph{$min\_pos$} et la position 
du maximun \emph{$max\_pos$}.
\par
Ainsi soient:
$a=\min(min\_pos, max\_pos) \land b=\max(min\_pos, max\_pos)$
\par
$somme=\sum_{i=a}^{b}(T[i]) \land 0 \le a \le b < N$

\subsection{Objets Utilisés:}

\begingroup
$min\_pos$: Contient l'indice de $\min{T}$
\endgroup
\begin{lstlisting}
int min_pos;
\end{lstlisting}

\begingroup
$max\_pos$: Contient l'indice de $\max{T}$
\endgroup
\begin{lstlisting}
int max_pos;
\end{lstlisting}

\begingroup
$resultat$: Contient la somme des élément du tableau $\max{T}$
\endgroup
\begin{lstlisting}
int resultat;
\end{lstlisting}

\subsection{Sous-Problème:}
Vu que le l'objectif de la fonction est d'écrire un programme de complexité $O(N)$,
on aura un seul sous-problème qui va consister en:
\begin{itemize}
    \item Determiner la position du maximun
    \item Determiner la position du minimun
    \item Calculer la somme des éléments du tableau entre le maximun et le minimum
\end{itemize}

% !TEX root = ./main.tex
%%%%%%%%%%%%%%%%%%%%%%%%%%%%%%%%%%%%%%%%%%%%%%%%%%%%%%%%%%%%%%%%%%%%%%%%%%%%%%%%%%%%%%%%%%
% Dans cette section, spécifiez formellement chacun des sous-problèmes.                  %
%%%%%%%%%%%%%%%%%%%%%%%%%%%%%%%%%%%%%%%%%%%%%%%%%%%%%%%%%%%%%%%%%%%%%%%%%%%%%%%%%%%%%%%%%%
\section{Specifications}\label{specifications}
%%%%%%%%%%%%%%%%%%%%%%%%

\begingroup
/**
\par
\quad* @Pre: $N=N_0>0 \land T = T_0$
\par
\quad* @post: $T=T_0 \land 0 \le min\_pos \le max\_pos \le N-1 \land T[min\_pos]=\min{T} \land T[max\_pos]=\max{T}$
\par
*/
\endgroup
\begin{lstlisting}
    int somme(int *T, int N,  int *min_pos, int *max_pos);
\end{lstlisting}

\section{Invariants}\label{invariants}

\centering
\includegraphics[scale=0.8]{invariant.png}
\setlength{\marginparwidth}{10pt}
\begin{quote}
    \centering
    Figure 1: Invariant graphique
\end{quote}
\vspace{1.2cm}

Invariant formel
\par
\vspace{0.3cm}

$INV \overset{}{\equiv} N>0$
$\land T=T_0 \land$
$0 \leq min\_pos \leq$
$max\_pos \leq i \leq N-1$
$\land T[min\_pos]=\min T$
$\land T[max\_pos]=\max T$

\vspace{1.2cm}
\raggedright



% !TEX root = ./main.tex
%%%%%%%%%%%%%%%%%%%%%%%%%%%%%%%%%%%%%%%%%%%%%%%%%%%%%%%%%%%%%%%%%%%%%%%%%%%%%%%%%%%%%%%%%%
% Dans cette section, il est demandé d'appliquer l'approche constructive pour la         %
% construction de votre code.                                                            %
%                                                                                        %
% Pour chaque Sous-Problème (une sous-section/SP):                                       %
% - {Pré} INIT {INV}                                                                     %
% - déterminer le Critère d'Arrêt (et donc le Gardien de Boucle)                         %
% - {INV & B} ITER {INV}                                                                 %
% - {INV & !B} END {Post}                                                                %
% - Fonction de Terminaison (pensez à justifier sur base de l'Invariant Graphique)       %
% (une sous-sous-section/tiret)                                                          %
%%%%%%%%%%%%%%%%%%%%%%%%%%%%%%%%%%%%%%%%%%%%%%%%%%%%%%%%%%%%%%%%%%%%%%%%%%%%%%%%%%%%%%%%%%
\section{Approche Constructive}
%%%%%%%%%%%%%%%%%%%%%%%%%%%%%%%%

\begin{lstlisting}[caption={somme.c}]
//$N=N_0>0 \land T = T_0$
int somme(int *T, int N,  int *min_pos, int *max_pos)
{
    //$N=N_0>0 \land T = T_0$
    assert(N>0 && T != NULL && min_pos != NULL && max_pos != NULL);
    //$N=N_0>0 \land T = T_0$

    int resultat, tmp1, tmp2, i;
    //$N=N_0>0 \land T = T_0$

    if (N==1) 
        //$N=N_0>0 \land T = T_0 \land T[min\_pos]=\min{T} \land T[max\_pos]=\max{T}$
        return T[*min_pos = *max_pos = 0];
    
    //$N=N_0>0 \land T = T_0$
    if (T[0] > T[1])
        *min_pos = 1 + (*max_pos = 0);
        //$N=N_0>0 \land T = T_0 \land T[min\_pos]=\min{T} \land T[max\_pos]=\max{T}$
    else
        *max_pos = 1 + (*min_pos = 0);
        //$N=N_0>0 \land T = T_0 \land T[min\_pos]=\min{T} \land T[max\_pos]=\max{T}$
    
    //$\rightarrow N=N_0>0 \land T = T_0 \land T[min\_pos]=\min{T} \land T[max\_pos]=\max{T}$

    resultat = T[0] + T[1];
    //$N=N_0>0 \land T = T_0 \land T[min\_pos]=\min{T} \land T[max\_pos]=\max{T} \land resultat = \sum_{k=min\_pos}^{max\_pos}(T[k])$
    tmp1 = T[0];
    //$tmp1=\sum_{}^{\min(min\_pos, max\_pos)}(T[.])$
    //$N=N_0>0 \land T = T_0 \land T[min\_pos]=\min{T} \land T[max\_pos]=\max{T} \land resultat = \sum_{k=min\_pos}^{max\_pos}(T[k])$
    tmp2 = 0;
    //$tmp2=\sum_{\max(min\_pos, max\_pos)}^{N-1}(T[.])$
    //$N=N_0>0 \land T = T_0 \land T[min\_pos]=\min{T} \land T[max\_pos]=\max{T} \land resultat = \sum_{k=min\_pos}^{max\_pos}(T[k])$
    i = 2;
    //$INV \overset{}{\equiv} N=N_0>0 \land T = T_0 \land T[min\_pos]=\min{T} \land T[max\_pos]=\max{T} $
    //$\land resultat = \sum_{k=min\_pos}^{max\_pos}(T[k]) \land 1 \le i \le N$

    while(i<N) 
    {
        //$INV \overset{}{\equiv} N=N_0>0 \land T = T_0 \land T[min\_pos]=\min{T} \land T[max\_pos]=\max{T} $
        //$\land resultat = \sum_{k=min\_pos}^{max\_pos}(T[k]) \land 1 \le i \le N-1$
        if(T[i]<T[*min_pos])
        {
            if (*min_pos<*max_pos) 
                resultat -= tmp1;
                //$resultat = \sum_{k=min\_pos}^{max\_pos}(T[k]$
            
            //$T[min\_pos]=\min{T}$
            *min_pos = i;
            //$T[min\_pos]=\min{T}$
            tmp1 = resultat;
            //$tmp1=\sum_{}^{\min(min\_pos, max\_pos)}(T[.])$
            //$tmp2=\sum_{\max(min\_pos, max\_pos)}^{i-1}(T[.])$
            tmp2 = 0;
            //$tmp2=\sum_{\max(min\_pos, max\_pos)}^{i}(T[.])$
        }
        else if(T[i]>T[*max_pos])
        {
            if (*max_pos<*min_pos) 
                resultat -= tmp1;
                //$resultat = \sum_{k=min\_pos}^{max\_pos}(T[k]$
            
            //$T[max\_pos]=\max{T}$
            *max_pos = i;
            //$T[max\_pos]=\max{T}$
            tmp1 = resultat;
            //$tmp1=\sum_{}^{\min(min\_pos, max\_pos)}(T[.])$
            //$tmp2=\sum_{\max(min\_pos, max\_pos)}^{i-1}(T[.])$
            tmp2 = 0;
            //$tmp2=\sum_{\max(min\_pos, max\_pos)}^{i}(T[.])$
        }
        else 
        {
            //$tmp2=\sum_{\max(min\_pos, max\_pos)}^{i-1}(T[.])$
            tmp2 += T[i];
            //$tmp2=\sum_{\max(min\_pos, max\_pos)}^{i}(T[.])$
        }
        
        resultat += T[i];
        //$resultat = \sum_{k=min\_pos}^{max\_pos}(T[k]$
        //$1 \le i \le N-1$
        i++;
        //$1 \le i \le N$
    }

    //$INV \overset{}{\equiv} N=N_0>0 \land T = T_0 \land T[min\_pos]=\min{T} \land T[max\_pos]=\max{T} $
    //$\land resultat = \sum_{k=max\_pos}^{N-1}(T[k]) \land 1 \le i \le N$

    return resultat - tmp2;
    //$T=T_0 \land 0 \le min\_pos \le max\_pos \le N-1 \land T[min\_pos]=\min{T} \land T[max\_pos]=\max{T}$
}
//$T=T_0 \land 0 \le min\_pos \le max\_pos \le N-1 \land T[min\_pos]=\min{T} \land T[max\_pos]=\max{T}$
\end{lstlisting}

%Il est possible de faire référence à la ligne \ref{code:ret} de l'extrait de code.


% !TEX root = ./main.tex
%%%%%%%%%%%%%%%%%%%%%%%%%%%%%%%%%%%%%%%%%%%%%%%%%%%%%%%%%%%%%%%%%%%%%%%%%%%%%%%%%%%%%%%%%%
% Dans cette section, indiquez le code complet (sans assertions intermédiaires) de votre %
% solution                                                                               %
%%%%%%%%%%%%%%%%%%%%%%%%%%%%%%%%%%%%%%%%%%%%%%%%%%%%%%%%%%%%%%%%%%%%%%%%%%%%%%%%%%%%%%%%%%
\section{Code Complet}\label{code}
%%%%%%%%%%%%%%%%%%%%%%%

\begin{lstlisting}[caption={somme.h}]
/**
 * \file somme.h
 * \brief Header pour la somme min-max d'un tableau
 * \author HEUBA BATOMEN Franck Duval, Bilali Assalni
 * \version 0.1
 * \date 03/04/2023
 *
*/

#ifndef __SOMME__
#define __SOMME__

/*
 * @pre: N=N0>0 && T = T0
 * @post: T=T0 && N = N0 && T[min_pos]=min(T) && T[max_pos]=max(T) 
 * && somme = min(min_pos, max_pos) + T[min(min_pos, max_pos) + 1] 
 * + ... + max(min_pos, max_pos)  
*/
int somme(int *T, int N,  int *min_pos, int *max_pos);

#endif
\end{lstlisting}

\begin{lstlisting}[caption={somme.c}]
int somme(int *T, int N,  int *min_pos, int *max_pos)
{
    assert(N>0 && T != NULL && min_pos != NULL && max_pos != NULL);

    int resultat, tmp1, tmp2, i;

    if (N==1) 
        return T[*min_pos = *max_pos = 0];

    if (T[0] > T[1])
        *min_pos = 1 + (*max_pos = 0);
    else
        *max_pos = 1 + (*min_pos = 0);

    resultat = T[0] + T[1];
    tmp1 = T[0];
    tmp2 = 0;
    i = 2;

    while(i<N) 
    {
        if(T[i]<T[*min_pos])
        {
            if (*min_pos<*max_pos) 
                resultat -= tmp1;
                
            *min_pos = i;
            tmp1 = resultat;
            tmp2 = 0;
        }
        else if(T[i]>T[*max_pos])
        {
            if (*max_pos<*min_pos) 
                resultat -= tmp1;
            
            *max_pos = i;
            tmp1 = resultat;
            tmp2 = 0;
        }
        else 
        {
            tmp2 += T[i];
        }
        
        resultat += T[i];
        i++;
    }

    return resultat - tmp2;
}
\end{lstlisting}

% !TEX root = ./main.tex
%%%%%%%%%%%%%%%%%%%%%%%%%%%%%%%%%%%%%%%%%%%%%%%%%%%%%%%%%%%%%%%%%%%%%%%%%%%%%%%%%%%%%%%%%%
% Dans cette section, vous devez étudier complètement la complexité de votre code.       %
% Soyez le plus formel (i.e., mathématique) possible.                                    %
%%%%%%%%%%%%%%%%%%%%%%%%%%%%%%%%%%%%%%%%%%%%%%%%%%%%%%%%%%%%%%%%%%%%%%%%%%%%%%%%%%%%%%%%%%
\section{Complexité}\label{complexite}
%%%%%%%%%%%%%%%%%%%%


% !TEX root = ./main.tex
%%%%%%%%%%%%%%%%%%%%%%%%%%%%%%%%%%%%%%%%%%%%%%%%%%%%%%%%%%%%%%%%%%%%%%%%%%%%%%%%%%%%%%%%%%
% Rédigez ici la conclusion de votre rapport.                                            %
%%%%%%%%%%%%%%%%%%%%%%%%%%%%%%%%%%%%%%%%%%%%%%%%%%%%%%%%%%%%%%%%%%%%%%%%%%%%%%%%%%%%%%%%%%
\section{Conclusion}\label{conclusion}
%%%%%%%%%%%%%%%%%%%%%


%%%%%%%%%%%%%%%%%%%% FIN DE LA ZONE PROTÉGÉE %%%%%%%%%%%%%%%%%%%%

\end{document}
