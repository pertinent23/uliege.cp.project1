% !TEX root = ./main.tex
%%%%%%%%%%%%%%%%%%%%%%%%%%%%%%%%%%%%%%%%%%%%%%%%%%%%%%%%%%%%%%%%%%%%%%%%%%%%%%%%%%%%%%%%%%
% Dans cette section, vous devez étudier complètement la complexité de votre code.       %
% Soyez le plus formel (i.e., mathématique) possible.                                    %
%%%%%%%%%%%%%%%%%%%%%%%%%%%%%%%%%%%%%%%%%%%%%%%%%%%%%%%%%%%%%%%%%%%%%%%%%%%%%%%%%%%%%%%%%%
\section{Complexité}\label{complexite}
%%%%%%%%%%%%%%%%%%%%

\par
determinons la complexité de notre code:

\raggedright\par

On peut diviser notre code en 05 parties:
\begin{itemize}

\item De la ligne 1\textminus2: \textbf{$P_a$}
\begin{lstlisting}
assert(N>0 && T != NULL && min_pos != NULL && max_pos != NULL);
int resultat, tmp1, tmp2, i;
\end{lstlisting}

\item De la ligne 4\textminus9: \textbf{$P_b$}
\begin{lstlisting}
if (N==1) 
    return T[*min_pos = *max_pos = 0];
elif (T[0] > T[1])
    *min_pos = 1 + (*max_pos = 0);
else
    *max_pos = 1 + (*min_pos = 0);
\end{lstlisting}

\item De la ligne 11\textminus14: \textbf{$P_c$}
\begin{lstlisting}
resultat = T[0] + T[1];
tmp1 = T[0];
tmp2 = 0;
i = 2;
\end{lstlisting}

\item De la ligne 16\textminus43: \textbf{$P_d$}
\begin{lstlisting}
while(i<N) 
{
    if(T[i]<T[*min_pos])
    {
        if (*min_pos<*max_pos) 
            resultat -= tmp1;
            
        *min_pos = i;
        tmp1 = resultat;
        tmp2 = 0;
    }
    else if(T[i]>T[*max_pos])
    {
        if (*max_pos<*min_pos) 
            resultat -= tmp1;
        
        *max_pos = i;
        tmp1 = resultat;
        tmp2 = 0;
    }
    else 
    {
        tmp2 += T[i];
    }
    
    resultat += T[i];
    i++;
}
\end{lstlisting}

\item De la ligne 45: \textbf{$P_e$}
\begin{lstlisting}
return resultat - tmp2;
\end{lstlisting}

\end{itemize}

\subsection[c]{$P_a$.}

$T(.)=1$

\subsection[c]{$P_b$.}

Ici la complexité est égale au maximun des complexités de chaque branche c'est à dire: $T(.)=1$

\subsection[c]{$P_c$.}

Ici toutes les instructions sont élémenetaires, ainsi: $T(.)=1$

\subsection[c]{$P_d$.}

Ici le code peut être diviser en quatre partie:

\begin{itemize}

\item De la ligne 19 à 25: $P_d1$ 
\begin{lstlisting}
if(T[i]<T[*min_pos])
{
    if (*min_pos<*max_pos) 
        resultat -= tmp1;
        
    *min_pos = i;
    tmp1 = resultat;
    tmp2 = 0;
}
\end{lstlisting}

\item De la ligne 29 à 34: $P_d2$ 
\begin{lstlisting}
else if(T[i]>T[*max_pos])
{
    if (*max_pos<*min_pos) 
        resultat -= tmp1;
    
    *max_pos = i;
    tmp1 = resultat;
    tmp2 = 0;
}
\end{lstlisting}

\item De la ligne 38 à 38: $P_d3$ 
\begin{lstlisting}
else 
{
    tmp2 += T[i];
}
\end{lstlisting}

\item De la ligne 41 à 42: $P_d4$ 
\begin{lstlisting}
resultat += T[i];
i++;
\end{lstlisting}

\end{itemize}

\subsubsection{$P_d1$}

Ici la complexité est $T(.)=1$

\subsubsection{$P_d2$}

Ici la complexité est $T(.)=1$


\subsubsection{$P_d3$}

Ici la complexité est $T(.)=1$


\subsubsection{$P_d4$}

Ici la complexité est $T(.)=1$


\par 
\par

Ainsi la complexité de notre programme est $T_d(.) = \sum_{i=1}^{N}(\max(T_d1(.), T_d2(.), T_d3(.)) + T_d4(.))$. 
\par
\par
Ainsi $T_d(.)=2N$

\subsection[c]{$P_e$.}

Enfin $T(.)=1$

\vspace{1cm}
$T(N) = T_a(.)+T_b(.)+T_c(.)+T_d(.) =3 + 2N$

\vspace{0.3cm}

$T \in O(N) \leftrightarrow\exists n_0 \in\emph{N}, c \in\emph{R}^+: \forall n>n_0:T(n)\leq c*n$

Ainsi pour $c \geq 4, T \in O(N)$.
\par
\vspace{0.3cm} 
La complexite de notre code est donc de: \textbf{$O(N)$}