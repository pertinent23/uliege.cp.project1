% !TEX root = ./main.tex
%%%%%%%%%%%%%%%%%%%%%%%%%%%%%%%%%%%%%%%%%%%%%%%%%%%%%%%%%%%%%%%%%%%%%%%%%%%%%%%%%%%%%%%%%%
% Rédigez ici la conclusion de votre rapport.                                            %
%%%%%%%%%%%%%%%%%%%%%%%%%%%%%%%%%%%%%%%%%%%%%%%%%%%%%%%%%%%%%%%%%%%%%%%%%%%%%%%%%%%%%%%%%%
\section{Conclusion}\label{conclusion}
%%%%%%%%%%%%%%%%%%%%%


\begin{flushleft}
\begingroup
\quad\quad Arrivé à l'épilogue de notre travail, il a été question de mener à bien un projet en respectant,
la méthodologie de développement.
\endgroup
\par 
\begingroup
\quad\quad Ainsi dans un premier temps, nous avons formalisez le problème 
et donnez une définition du problème. Ensuite nous avons données les spécifications à partie desquelles
nous avons construit l'Invariant graphique et ainsi nous avons déterminé l'invariant formel.
\endgroup
\par 
\begingroup
\quad\quad Ce dernier nous a permis d'écrire notre code et Ensuite de montrer qu'il est correcte; et enfin de 
determiner la complexité de notre code.
\endgroup
\end{flushleft}