% !TEX root = ./main.tex
%%%%%%%%%%%%%%%%%%%%%%%%%%%%%%%%%%%%%%%%%%%%%%%%%%%%%%%%%%%%%%%%%%%%%%%%%%%%%%%%%%%%%%%%%%
% Dans cette section, introduisez toutes les notations mathématiques que vous jugez      %
% utiles à la réalisation du projet.                                                     %
%%%%%%%%%%%%%%%%%%%%%%%%%%%%%%%%%%%%%%%%%%%%%%%%%%%%%%%%%%%%%%%%%%%%%%%%%%%%%%%%%%%%%%%%%%
\section{Formalisation du Problème}\label{formalisation}
%%%%%%%%%%%%%%%%%%%%%%%%%%%%%%%%%%%

\begin{table}[!h]
\centering
\begin{tabular}{l c}
Nom & Op \\
\hline
ET & $\land$ \\
OU & $\lor$ \\
Minimun d'un tableau & $\min$ \\
Maximun d'un tableau & $\max$ \\
Superieure & > \\
Inferieure & < \\
Superieure ou égale & $\ge$ \\
Inferieure ou égale & $\le$ \\
Somme d'éléments de 0 à k & $\sum_{0}^{k}$ \\
Quantification universelle & $\forall$ \\
Quantification existentielle & $\exists$ \\
\end{tabular}
\caption{Opérateurs les plus usuels en logique}
\end{table}